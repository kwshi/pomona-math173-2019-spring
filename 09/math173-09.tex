\documentclass{../homework}

\homework 9
\date{Tuesday 4/9}
\author{}
\coauthor{}

\begin{document}
\begin{problems}
\item[P.10.10] Let \(A \in \M_m\) and let \(B \in \M_n\).  Show that
  if \(\spec A \cap \spec B \ne \varnothing\), then there is a nonzero
  \(X \in \M_{m \times n}\) such that \(AX - Xb = 0\).  \textit{Hint}:
  You may find it helpful to investigate \(X = \vec x \vec y^*\), in
  which \((\lambda, \vec x)\) is an eigenpair of \(A\) and
  \((\lambda, \vec y)\) is an eigenpair of \(B^*\).

  \begin{solution}
    \begin{proof}

    \end{proof}
  \end{solution}

\item[P.10.22] Let \(A = A_1 \oplus A_2 \oplus \dots \oplus A_d\) and
  \(B = B_1 \oplus B_2 \oplus \dots \oplus B_d\) be \(n \times n\)
  matrices that are conformally partitioned and block diagonal.
  Suppose that \(\spec A_i \cap \spec B_j = \varnothing\) for all
  \(i \ne j\).  If \(C \in \M_n\) and \(AC = CB\), show that
  \(C = C_1 \oplus C_2 \oplus \dots \oplus C_d\) is block diagonal and
  conformal with \(A\) and \(B\).

  \begin{solution}
    \begin{proof}

    \end{proof}
  \end{solution}

\item[P.10.27] If \(A \in \M_5\) is diagonalizable and
  \(p_A(z) = (z-2)^3 (z-3)^2\), show that \(m_A(z) = (z-2)(z-3)\).

  \begin{solution}
    \begin{proof}

    \end{proof}
  \end{solution}

\item[P.10.29] If \(A \in \M_n\) is an involution, show that it is
  unitarily similar to a block matrix of the form
  \[
    \begin{bmatrix}
      I_k & X \\ 0 & -I_{n-k}
    \end{bmatrix},
    \quad X \in \M_{k \times (n-k)}.
  \]

  \begin{solution}
    \begin{proof}

    \end{proof}
  \end{solution}

\item[P.10.34] Let \(p(z) = z^2 + 4\).  Is there an
  \(A \in \M_3(\RR)\) with \(m_A(z) = p(z)\)?  Is there an
  \(A \in \M_2(\RR)\) with \(m_A(z) = p(z)\)?  Is there an
  \(A \in \M_3(\CC)\) with \(m_A(z) = p(z)\)?  In each case, provide a
  proof or an example.

  \begin{solution}

  \end{solution}

\item[P.10.38] Use the companion matrix to show that the Fundamental
  Theorem of Algebra is equivalent to the statement ``every square
  complex matrix has an eigenvalue.''

  \begin{solution}
    \begin{proof}

    \end{proof}
  \end{solution}

\item[P.11.1] Find the Jordan canonical forms of
  \[
    \begin{array}{ccc}
      \begin{bmatrix} 0&1&1&1\\0&0&1&1\\0&0&0&1\\0&0&0&0 \end{bmatrix}, &
      \begin{bmatrix} 0&1&1&0\\0&0&0&0\\0&0&0&1\\0&0&0&0 \end{bmatrix}, &
      \begin{bmatrix} 0&1&0&1\\0&0&0&0\\0&0&0&1\\0&0&0&0 \end{bmatrix}, \\[2em]
      \begin{bmatrix} 0&0&1&1\\0&0&0&1\\0&0&0&0\\0&0&0&0 \end{bmatrix}, &
      \begin{bmatrix} 0&0&1&1\\0&0&1&1\\0&0&0&0\\0&0&0&0 \end{bmatrix}, &
      \begin{bmatrix} 0&0&1&1\\0&0&0&1\\0&1&0&0\\0&0&0&0 \end{bmatrix}.
    \end{array}
  \]

  \begin{solution}
    \begin{proof}

    \end{proof}
  \end{solution}

\item[P.11.4] If \(A \in \M_5\), \((A-2I)^3 = 0\), and
  \((A-2I)^2 \ne 0\), what are the possible Jordan canonical forms for
  \(A\)?

  \begin{solution}

  \end{solution}
\end{problems}
\end{document}