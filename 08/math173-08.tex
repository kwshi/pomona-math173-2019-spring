\documentclass{../homework}

\homework 8
\date{Thursday 3/28}

\author{}
\coauthor{}

\begin{document}
\begin{problems}
\item[P.9.15] Let \(A, B \in \M_n\) and suppose that either \(A\) or
  \(B\) is invertible.  Show that \(AB\) is similar to \(BA\), and
  conclude that these two products have the [same] eigenvalues.

  \begin{solution}
    \begin{proof}

    \end{proof}
  \end{solution}

\item[P.9.16]
  \begin{enumerate}
  \item Consider the \(2\times2\) matrices \(A =
    \begin{bmatrix}
      0 & 1 \\ 0 & 0
    \end{bmatrix}
    \) and \(B =
    \begin{bmatrix}
      0 & 0 \\ 1 & 0
    \end{bmatrix}
    \).  Is \(AB\) similar to \(BA\)?  Do these two products have the
    same eigenvalues?

    \begin{solution}

    \end{solution}

  \item Answer the same questions for
    \(A = \begin{bmatrix} 0 & 1 \\ 0 & 0 \end{bmatrix}\) and
    \(B = \begin{bmatrix} 1 & 0 \\ 0 & 0 \end{bmatrix}\).

    \begin{solution}

    \end{solution}
  \end{enumerate}
\item[P.9.18] Use the vectors \(\vec e\) and \(\vec r\) in Example
  9.7.9.
  \begin{bookexample}[9.7.9]
    Let \(A\) be the matrix (9.7.8) and observe that
    \[
      A + A^\transpose = \begin{bmatrix} i+j \end{bmatrix}
      =
      \begin{bmatrix}
        2 & 3 & 4 & \cdots & n+1 \\
        3 & 4 & 5 & \cdots & n+2 \\
        4 & 5 & 6 & \cdots & n+3 \\
        \vdots & \vdots & \vdots & \ddots & \vdots \\
        n+1 & n+2 & n+3 & \cdots & 2n
      \end{bmatrix}.
      \tag{9.7.10}
    \]
    Now use the presentation of a matrix product in (3.1.19) to wrie
    \[
      A + A^\transpose = \vec r \vec e^\transpose
      + \vec e \vec r^\transpose = XY^\transpose,
    \]
    in which
    \(X = \begin{bmatrix} \vec r & \vec e \end{bmatrix} \in \M_{n
      \times 2}\) and
    \(Y = \begin{bmatrix} \vec e & \vec r \end{bmatrix} \in \M_{n
      \times 2}\).  Theorem 9.7.2 tells us that the \(n\) eigenvalues
    of \(XY^\transpose\) are the two eigenvalues of
    \[
      Y^\transpose X =
      \begin{bmatrix}
        \vec e^\transpose \vec r & \vec e^\transpose \vec e \\
        \vec r^\transpose \vec r & \vec r^\transpose \vec e \\
      \end{bmatrix}
      \tag{9.7.11}
    \]
    together with \(n-2\) zeros; see P.9.17.
  \end{bookexample}

  \begin{book}
    \[
      A = \vec r \vec e^\transpose =
      \begin{bmatrix}
        1 & 1 & \cdots & 1 \\
        2 & 2 & \cdots & 2 \\
        \vdots & \vdots & \ddots & \vdots \\
        n & n & \cdots & n
      \end{bmatrix}
      \tag{9.7.8}
    \]
  \end{book}

  \begin{booktheorem}[9.7.2]
    Suppose that \(A \in \M_{m \times n}\), \(B \in \M_{n \times m}\),
    and \(n \ge m\).
    \begin{enumerate}
    \item The nonzero eigenvalues of \(AB \in \M_m\) and
      \(BA \in M_n\) are the same, with the same algebraic
      multiplicities.
    \item If \(0\) is an eigenvalue of \(AB\) with algebraic
      multiplicity \(k \ge 0\), then \(0\) is an eigenvalue of \(BA\)
      with algebraic multiplicity \(k+n-m\).
    \item If \(m=n\), then the eigenvalues of \(AB\) and \(BA\) are
      the same, with the same algebraic multiplicities.
    \end{enumerate}
  \end{booktheorem}

  \begin{enumerate}
  \item Verify that
    \[
      A = \begin{bmatrix} i-j \end{bmatrix} =
      \begin{bmatrix}
        0 & -1 & -2 & \cdots & -n+1 \\
        1 & 0 & -1 & \cdots & -n+2 \\
        2 & 1 & 0 & \cdots & -n+3 \\
        \vdots & \vdots & \vdots & \ddots & \vdots \\
        n-1 & n-2 & n-3 & \cdots & 0
      \end{bmatrix}
      = \vec r \vec e^\transpose - \vec e \vec r^\transpose
      = ZY^\transpose,
    \]
    in which \(Y = \begin{bmatrix} \vec e & \vec r \end{bmatrix}\) and
    \(Z = \begin{bmatrix} \vec r & -\vec e \end{bmatrix}\).

    \begin{solution}

    \end{solution}

  \item Show that the eigenvalues of \(A\) are the two eigenvalues of
    \[
      Y^\transpose Z =
      \begin{bmatrix}
        \vec e^\transpose \vec r & - \vec e^\transpose \vec e \\
        \vec r^\transpose \vec r & - \vec r^\transpose \vec e \\
      \end{bmatrix}
      =
      \begin{bmatrix}
        \frac 1 2 n(n+1) & -n \\
        \frac 1 6 n(n+1)(2n+1) & -\frac 1 2 n(n+1)
      \end{bmatrix},
    \]
    together with \(n-2\) zeros.

    \begin{solution}
      \begin{proof}

      \end{proof}
    \end{solution}

  \item Show that the discriminant of \(Y^\transpose Z\) is negative
    (see P.9.8) and explain what this implies about the eigenvalues.

    \begin{solution}
      \begin{proof}

      \end{proof}
    \end{solution}

  \item Show that the eigenvalues of \(Y^\transpose Z\) are
    \[
      \pm i \frac n 2 \sqrt{\frac{n^2-1}{3}}.
    \]

    \begin{solution}
      \begin{proof}

      \end{proof}
    \end{solution}
  \end{enumerate}

\item[P.9.25] If \(A \in \M_n(\RR)\) and \(n\) is odd, show that \(A\)
  has at least one real eigenvalue.

  \begin{solution}
    \begin{proof}

    \end{proof}
  \end{solution}

\item[P.10.2] Let \(A \in \M_n\).  Prove in two ways that \(A\) is
  nilpotent if and only if \(\spec A = \set 0\).
  \begin{enumerate}
  \item Use Theorem 10.1.1 and consider powers of a strictly upper
    triangular matrix.

    \begin{solution}
      \begin{proof}

      \end{proof}
    \end{solution}

  \item Use Theorems 10.2.1 and 8.3.3.

    \begin{solution}
      \begin{proof}

      \end{proof}
    \end{solution}
  \end{enumerate}

\item[P.10.3] Let \(A \in \M_n\).  Show that the following statements
  are equivalent:
  \begin{enumerate}
  \item \(A\) is nilpotent.
  \item \(A\) is unitarily similar to a strictly upper triangular
    matrix.
  \item \(A\) is similar to a strictly upper triangular matrix.
  \end{enumerate}

  \begin{solution}
    \begin{proof}

    \end{proof}
  \end{solution}

\item[P.10.4] Suppose that an upper triangular matrix \(T \in \M_n\)
  has \(\nu\) nonzero diagonal entries.  Show that \(\rank T \ge \nu\)
  and give an example for which \(\rank T > \nu\).

  \begin{solution}
    \begin{proof}

    \end{proof}
  \end{solution}

\item[P.10.5] Suppose that \(A \in \M_n\) has \(\nu\) nonzero
  eigenvalues.  Explain why \(\rank A \ge \nu\) and give an example
  for which \(\rank A > \nu\).

  \begin{solution}

  \end{solution}
\end{problems}
\end{document}
