\documentclass{../homework}

\homework{7}
\date{Tuesday 3/12}

\author{}
\coauthor{}

\begin{document}
\begin{problems}
\item[P.8.8] Let \(n \ge 2\), let \(c \in C\), and consider
  \(A_c \in \M_n\), whose off-diagonal entries are all \(1\), and
  whose diagonal entries are \(c\).  Sketch the Ger\v sgorin region
  \(G(A_c)\).  Why is \(A_c\) invertible if \(\abs c > n - 1\)?  Show
  that \(A_{1-n}\) is not invertible.

  \begin{solution}

  \end{solution}

\item[P.8.9*] Let \(E = \vec e \vec e^\transpose \in \M_n\).
  \begin{enumerate}
  \item Why is \((n, \vec e)\) an eigenpair of \(E\)?

    \begin{solution}

    \end{solution}

  \item If \(\vec v\) is nonzero and orthogonal to \(\vec e\), why is
    \((0, \vec v)\) an eigenpair of \(E\)?

    \begin{solution}

    \end{solution}

  \item Let \(p(z) = z^2 - nz\) and show that \(p(E) = 0\).

    \begin{solution}

    \end{solution}

  \item Use Theorem 8.3.3 to explain why \(0\) and \(n\) are the only
    eigenvalues of \(E\).  What are their respective geometric
    multiplicities?
    \begin{booktheorem}[8.3.3]
      Let \(A \in \M_n\) and let \(p\) be a polynomial that
      annihilates \(A\).
      \begin{enumerate}
      \item Every eigenvalue of \(A\) is a root of \(p(z) = 0\).
      \item \(A\) has finitely many different eigenvalues.
      \end{enumerate}
    \end{booktheorem}

    \begin{solution}

    \end{solution}
  \end{enumerate}

\item[P.8.10] Consider \(A =
  \begin{bmatrix}
    a_{ij}
  \end{bmatrix}
  \in \M_n\), in which \(a_{ii} = a\) for each \(i = 1, 2, \dots, n\)
  and \(a_{ij} = b\) if \(i \ne j\).  Find the eigenvalues of \(A\) if
  their geometric multiplicities.  \textit{Hint}: Use the preceding
  problem.

  \begin{solution}

  \end{solution}

\item[P.8.15] Let \(A \in \M_n\), \(B \in \M_m\), and
  \(C = A \oplus B \in \M_{n+m}\).  Use eigenpairs to show the
  following:
  \begin{enumerate}
  \item If \(\lambda\) is an eigenvalue of \(A\), then it is an
    eigenvalue of \(C\).

    \begin{solution}
      \begin{proof}

      \end{proof}
    \end{solution}

  \item If \(\lambda\) is an eigenvalue of \(B\), then it is an
    eigenvalue of \(C\).

    \begin{solution}
      \begin{proof}

      \end{proof}
    \end{solution}

  \item If \(\lambda\) is an eigenvalue of \(C\), then it is an
    eigenvalue of either \(A\) or \(B\).

    \begin{solution}
      \begin{proof}

      \end{proof}
    \end{solution}
  \end{enumerate}

\item[P.8.27] The \textit{Volterra operator} is the linear operator
  \(T \colon C[0, 1] \to C[0, 1]\) defined by
  \((Tf)(t) = \int_0^t f(s) \dif s\).  The pair \((\lambda, f)\) is an
  \emph{eigenpair} of \(T\) if \(\lambda \in C\), \(f \in C[0, 1]\) is
  not the zero function, and \((Tf)(t) = \lambda f(t)\) for all
  \(t \in [0, 1]\).
  \begin{enumerate}
  \item Show that \(0\) is not an eigenvalue of \(T\).

    \begin{solution}
      \begin{proof}

      \end{proof}
    \end{solution}

  \item Find the eigenpairs of the Volterra operator.  Compare your
    results with the situation for eigenvalues of a matrix.
    \textit{Hint}: Consider the equation \(Tf = \lambda f\) and use
    the fundamental theorem of calculus.

    \begin{solution}

    \end{solution}
  \end{enumerate}

\item[P.8.28] Suppose that \((\lambda, \vec x)\) and \((\mu, \vec y)\)
  are eigenpairs of \(A \in \M_n\) and \(B \in M_m\), respectively.
  Show that \((\lambda \mu, \vec x \otimes \vec y)\) is an eigenpair
  of \(A \otimes B \in \M_{nm}\) and
  \((\lambda+\mu, \vec x \otimes \vec y)\) is an eigenpair of
  \(\Paren{A \otimes I_m} + \Paren{I_n \otimes B} \in \M_{nm}\).

  \begin{solution}
    \begin{proof}

    \end{proof}
  \end{solution}

\item[P.9.3] Let \(A, B \in \M_n\).
  \begin{enumerate}
  \item If \(\spec A = \spec B\), do \(A\) and \(B\) have the same
    characteristic polynomials?  Why?

    \begin{solution}

    \end{solution}

  \item If \(A\) and \(B\) have the same characteristic polynomials,
    is \(\spec A = \spec B\)?  Why?

    \begin{solution}

    \end{solution}
  \end{enumerate}

\item[P.9.6] Suppose tht \(A \in \M_n\) is diagonalizable.
  \begin{enumerate}
  \item Show that \(\rank A\) is equal to the number of its nonzero
    eigenvalues (including multiplicities).

    \begin{solution}
      \begin{proof}

      \end{proof}
    \end{solution}

  \item Consider the matrix
    \[
      B =
      \begin{bmatrix}
        0 & 1 \\
        0 & 0
      \end{bmatrix}.
    \]
    What is its rank?  How many nonzero eigenvalues does it have?  Is
    it diagonalizable?

    \begin{solution}

    \end{solution}
  \end{enumerate}
\end{problems}
\end{document}