\documentclass{../homework}

\homework{5}
\date{Tuesday 2/26}

\author{}
\coauthor{}

\begin{document}
\begin{problems}
\item[P.7.15] Let \(P \in \M_n\) be an orthogonal projection.  Show
  that \(\tr P = \dim \col P\).

  \begin{solution}

  \end{solution}

\item[P.7.17] Let \(\Vspace\) be a finite-dimensional \(\FF\)--inner
  product space and let \(P, Q \in L(\Vspace)\) be orthogonal
  projections.  Show that the following are equivalent:
  \begin{enumerate}
  \item \(\ran P \perp \ran Q\).
  \item \(\ran Q \subseteq \ker P\).
  \item \(PQ = 0\).
  \item \(PQ + QP = 0\).
  \item \(P+Q\) is an orthogonal projection.
  \end{enumerate}

  \begin{solution}

  \end{solution}

\item[P.7.20] Let \(\Vspace\) be a finite-dimensional \(\FF\)--inner
  product space and let \(P \in L(\Vspace)\) be idempotent.  Show that
  if \(\ker P \subset (\ran P)^\perp\), then \(P\) is the orthogonal
  projection onto \(\ran P\).

  \begin{solution}

  \end{solution}

\item[P.7.21] Let \(\Vspace\) be a finite-dimensional \(\FF\)--inner
  product space and let \(P \in L(\Vspace)\) be idempotent.  Show that
  \(\norm{P \vec v} \le \norm{\vec v}\) for every \(\vec v \in V\) if
  and only if \(P\) is an orthogonal projection.  \textit{Hint}: use
  the preceding problem and P.4.10.
  \begin{book}
    \begin{problems}
    \item[P.4.10] Let \(\Vspace\) be an \(\FF\)--inner product space
      and let \(\vec u, \vec v \in \Vspace\).  Prove that \(\vec u\)
      and \(\vec v\) are orthogonal if and only if
      \(\norm{\vec v} \le \norm{c \vec u + \vec v}\) for all
      \(c \in \FF\).  Be sure your proof covers both cases
      \(\FF = \RR\) and \(\FF = \CC\).  Draw a diagram illustrating
      what this means if \(\Vspace = \RR^2\).
    \end{problems}
    \begin{center}
      \textcolor{gray}{(assigned in Homework 2)}
    \end{center}
  \end{book}

  \begin{solution}

  \end{solution}

\item[P.7.22] Let
  \(P = \begin{bmatrix} p_{ij} \end{bmatrix} \in \M_n\) be idempotent
  and such that \(p_{11} = p_{22} = \dots = p_{nn} = 0\).  Prove that
  \(P = 0\).

  \begin{solution}

  \end{solution}

\item[P.7.23] Let \(A \in \M_{m \times n}\), suppose that
  \(\rank A = n\), and let \(A = QR\) be a \(QR\) factorization.  If
  \(\vec y \in \col A\), show that \(\vec x_0 = R^{-1} Q^* \vec y\) is
  the unique solution of the linear system \(A \vec x = \vec y\).

  \begin{solution}

  \end{solution}

\item[P.7.24] Let \(A \in \M_{m \times n}\), suppose that
  \(\rank A = n\), and let \(A = QR\) be a \(QR\) factorization.  Show
  that \(A \Paren{A^* A}^{-1} A^* = Q Q^*\).

  \begin{solution}

  \end{solution}

\item[P.7.25] Let \(\Vspace\) be an inner product space and let
  \(\vec u_1, \vec u_2, \dots, \vec u_n \in V\) be linearly
  independent.  Show that \(G(\vec u_1, \vec u_2, \dots, \vec u_n)\)
  is invertible.  \textit{Hint}:
  \(\vec x^* G (\vec u_1, \vec u_2, \dots, \vec u_n) \vec x =
  \Norm{\sum_{i=1}^n x_i \vec u_i}^2\).

  \begin{solution}

  \end{solution}
\end{problems}
\end{document}