\documentclass{../homework}

\homework{12}
\date{Tuesday 4/30}

\author{}
\coauthor{}

\begin{document}
\begin{problems}
\item[P.13.15] Prove the Cauchy-Schwarz inequality in \(\FF^n\) by
  computing \(\det A^* A\), in which
  \(A = \begin{bmatrix} \vec x & \vec y \end{bmatrix}\) and
  \(\vec x, \vec y \in \FF^n\).

  \begin{solution}
    \begin{proof}

    \end{proof}
  \end{solution}

\item[P.13.22] Let
  \(A = \begin{bmatrix} a_{ij} \end{bmatrix} \in \M_n\) be positive
  definite and let \(k \in \set{1, 2, \dots, n-1}\).  Show in the
  following two ways that the \(k \times k\) leading principal
  submatrix of \(A\) is positive definite.

  \begin{enumerate}
  \item Use the Cholesky factorization of \(A\).

    \begin{solution}
      \begin{proof}

      \end{proof}
    \end{solution}

  \item Consider \(\inner{A \vec x, \vec x}\) in which the last
    \(n - k\) entries of \(\vec x\) are zero.

    \begin{solution}
      \begin{proof}

      \end{proof}
    \end{solution}
  \end{enumerate}

\item[P.13.32] Let \(\Vspace\) be an \(\FF\)-inner product space and
  let \(\vec u_1, \vec u_2, \dots, \vec u_n \in \Vspace\).

  \begin{enumerate}
  \item Show that the Gram matrix
    \(G = \begin{bmatrix} \inner{\vec u_j, \vec u_i} \end{bmatrix} \in
    \M_n(\FF)\) is positive semidefinite.

    \begin{solution}
      \begin{proof}

      \end{proof}
    \end{solution}

  \item Show that \(G\) is positive definite if and only if
    \(\vec u_1, \vec u_2, \dots, \vec u_n\) are linearly independent.

    \begin{solution}
      \begin{proof}

      \end{proof}
    \end{solution}

  \item Show that the matrix
    \(\begin{bmatrix} (i + j - 1)^{-1} \end{bmatrix}\) in P.4.23 is
    positive definite.

    \begin{solution}
      \begin{proof}

      \end{proof}
    \end{solution}
  \end{enumerate}

\item[P.13.38] Let \(H \in \M_n\) be Hermitian and suppose that
  \(\spec H \subset [-1, 1]\).

  \begin{enumerate}
  \item Show that \(I - H^2\) is positive semidefinite.

    \begin{solution}
      \begin{proof}

      \end{proof}
    \end{solution}

  \item Let \(U = H + i(I - H^2)^{1/2}\).  Show that \(U\) is unitary
    and \(H = \frac{1}{2}(U + U^*)\).

    \begin{solution}
      \begin{proof}

      \end{proof}
    \end{solution}
  \end{enumerate}

\item[P.13.39] Let \(A \in \M_n\).  Use the Cartesian decomposition
  and the preceding problem to show that \(A\) is a linear combination
  of at most four unitary matrices.  See P.15.40 for a related result.

  \begin{solution}
    \begin{proof}

    \end{proof}
  \end{solution}

\item[P.13.51] If \(A \in M_m\) and \(B \in M_m\) are positive semidefinite,
  show that \(A \otimes B\) is positive semidefinite.

  \begin{solution}
    \begin{proof}

    \end{proof}
  \end{solution}

\item[P.13.53] Let \(A = [a_{ij}] \in \M_n\) and
  \(B = [b_{ij}] \in \M_n\).
  \begin{enumerate}
  \item Show that \(A \circ B \in \M_n\) is a principal submatrix of
    \(A \otimes B \in \M_{n^2}\).

    \begin{solution}
      \begin{proof}

      \end{proof}
    \end{solution}

  \item Deduce Theorem 13.5.5 from (a) and P.13.51.

    \begin{solution}
      \begin{proof}

      \end{proof}
    \end{solution}
  \end{enumerate}

  \begin{solution}
  \end{solution}

\item[P.14.4] Give \(2 \times 2\) examples to show that a matrix can
  have:
  \begin{enumerate}
  \item equal singular values but distinct eigenvalues, or

    \begin{solution}

    \end{solution}

  \item equal eigenvalues but distinct singular values.
    \begin{solution}

    \end{solution}
  \end{enumerate}
\end{problems}
\end{document}